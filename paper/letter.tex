\documentclass[]{letter}

\usepackage{graphicx}
\usepackage{enumitem}

\usepackage[margin=1.25in]{geometry}

\hyphenation{man-u-script}

\signature{Eric Hunsberger\\Matthew Scott\\Chris Eliasmith}
\address{Centre for Theoretical Neuroscience\\200 University Ave. W.\\Waterloo, ON, Canada N2L 3G1}
\date{\today}

\begin{document}
\begin{letter}{Editorial Board\\Neural Computation}
\opening{Dear Editor:}

Please find enclosed a revised version of our paper
entitled ``The competing benefits of noise and heterogeneity in neural coding''.
In this revision, we address the further concerns of reviewer one.
We respond to each comment individually,
with the quoted comment presented in an indented paragraph,
followed by our response.

\ \newline
{\large \bf Reviewer~1}

\begin{quotation}
  The revised manuscript is substantially improved and may be of interest to the computational and experimental neuroscience community.  The authors have done an especially good job reviewing the literature, and the greater clarity with which the relationship between stochastic resonance and heterogeneity in the manuscript is described improves the readability greatly.  The authors should be commended for this.  In addition, eliminating some of the other components of the initial submission makes the conclusions of the manuscript much more compelling.  Addressing one major comment and the minor points of clarification below would help to drive home the results in this paper.

\vspace{1em}
\noindent Major point:
\vspace{0.5em}

The authors use the bias (the threshold of the neuron) as their metric for heterogeneity.  The biological motivation needs to be addressed more clearly.  Specifically, this form of diversity may be perceived as somewhat artificial, but if the authors could justify it with references or a few examples, this would go a long way.  An alternative way to generate diversity, at least in the leaky integrate and fire neuron, would be to vary the time constants (tau) for the cells would it not?  As the authors cite McCormick et al 1985, they should note that while the mean tau in that paper was 20 ms, the variance was +/-14 ms for regular spiking cells. Across neurons of different types the tau varies even more.  Or vary tau ref, which would correspond to differential expression of channels with varied kinetics that contribute to differences in diversity of refractory times.  Such a metric of diversity would be more realistic and distinguish this result from others wherein the use of the variable threshold has been employed, as the authors themselves concede with ``research into stochastic resonance with heterogeneity has focused on populations of binary threshold units, which fail to capture the dynamics of actual neurons''.  This is especially important to really distinguish the novelty of this work from other in the field of heterogeneity where a threshold diversity is used.
\end{quotation}

We have added a few sentences to both the introduction and discussion to better motivate the use of heterogeneity in the bias (threshold) as the metric for heterogeneity in our paper. We agree that other forms of diversity are important, but as we note in the discussion, using forms of heterogeneity that affect the maximum firing rate of the neurons would require a more complex decoding scheme than what we have used (see, for example, Eliasmith 2003). We therefore leave this as future work.


\begin{quotation}
\noindent Minor points:
\vspace{0.5em}

The authors should motivate the choice of the stimulus better.  In other examples, both experimental (see Mainen and Sejnowski, 1995) and theoretical (Galan et al 2008), the power spectrum of the input signal seems to be qualitatively different that the ones being used by the authors, or if this is a misunderstanding on my part, then the authors should clearly state this.   As the statistics of the input shape reliability (Galan et al 2008) and encoding fidelity (RR de Ruyter van Steveninck et al 1997), it would be important to ensure that the stimulus parameters match those seen in real neurons and that changing the stimulus parameters within other domains of plausibility does not affect their result.  Even if it does, it would be ok because it would highlight that different combinations of heterogeneity and noise that are optimal depending on the stimulus statistics.  Either way this should be considered.
\end{quotation}

We have re-run all simulations using the alpha function-filtered white noise used by Mainen and Sejnowski (1995) and Gal\'an et al. (2008). This significantly changed some results, specifically the information landscape for FHN neurons, so we revised our discussion to reflect this. The results for FHN and LIF neurons are actually much more similar now than previously.

\begin{quotation}
In figure 5, the y axis (64 different neurons) should be labelled as such to ensure that neurons are not confused with trials.

Also for figure 5, it would be helpful to readers to see the stimulus that drives the neurons.  Are they the same in all conditions?
\end{quotation}

We have made both these changes; they both improve the figure. The stimulus now appears at the top of the figure, and is the same for all conditions.

\begin{quotation}
In most of the figures, the authors report I (mutual information) in units of bits.   Given the increase in firing rate in the high noise case (Fig. 5), it may be worth reporting information in bits/spike - a qualitative look at figures 2 and 5 suggests that the central result should not be affected, and the shape of the landscape in figure 2 may be even become more peaky, further highlighting the result of the paper.  If the authors are reporting their result in bits per spike and I missed this, then they should update the axes on the figures to clarify this.
\end{quotation}

We ran simulations measuring the information per spike, and found that the results are significantly different than when measuring the information content of the system, especially for FHN neurons. We have made significant revisions to the manuscript to discuss this new result.

\begin{quotation}
The authors should also make note that the information metric is calculated on the summing neuron (Stocks and Mannella, 2001).  They do so in the methods, but a reminder would be useful for readers as a schematic in figure 1 - it could be as simple as a box diagram with stimulus going into 64 neurons, then going into a box for the decoder neuron.  This would also help clarify and distinguish their information calculation method and subsequent result from those of Osborne et. al., 2008 and Padmanabhan and Urban, 2010 where the information metric is calculated across the population spike train.
\end{quotation}

We added an introductory figure depicting the experimental setup.

\begin{quotation}
A motivation should be given for the choice of 64 neurons and the authors should state in the text that the population size did not affect their central result (assuming of course that it does not).  Population size effects can be significant when talking about information content, and while the Osborne at al 2008 result suggests that information grows with population size, it would be important to show that this holds true in the formulation selected by the authors.  Simply put, does the landscape in Fig. 2 change if the population is 5 neurons vs. 50 neurons vs. 100 cells?
\end{quotation}

We ran simulations with other numbers of neurons, and found that it does not affect the qualitative shape of the information landscapes. As might be expected, using fewer neurons lowers the overall information content of the system. We added two sentences to this regard to the experiment description at the beginning of the results section.

{\large \bf Reviewer~2}
\begin{quotation}
  The manuscript is significantly improved and the authors have addressed all my major concerns.
\end{quotation}


{\large \bf Additional Changes}

Previously, we had considered the FHN neuron model to be spiking whenever the membrane voltage was greater than zero. This allowed information not only to be transmitted in the occurrence and timing of spikes, but also in their amplitude and exact structure. This particularly became a problem at large levels of heterogeneity, which were able to push the model into a regime where it spent long periods with a positive membrane voltage, allowing direct transmission of the input signal to the output via these neurons, and ultimately resulting in inflated information transmission rates. We therefore revised our method of spike transmission for the FHN neuron model, so that spikes are transmitted as instantaneous stereotyped events, as with the LIF neuron model. This resulted in significantly different, and in our opinion more sensible, results for the FHN neuron model (see Figure 3).

We hope that the changes detailed here address the concerns of the reviewers. We would like to thank them for their time, and for their insightful comments which have helped us improve the quality of the manuscript.

\closing{Sincerely,}

\end{letter}
\end{document}
