\documentclass[]{letter}

\usepackage{graphicx}

\usepackage[margin=1.375in]{geometry}

%% \topmargin 0.0cm
%% \oddsidemargin 0.0cm
%% \evensidemargin 0.0cm
%% \textwidth 5.75in
%% \textheight 9.25in

%% \usepackage{hyperref}

\signature{Eric G. Hunsberger}
\address{Centre for Theoretical Neuroscience\\200 University Ave. W.\\Waterloo, ON, Canada N2L 3G1}
\date{\today}

\begin{document}
\begin{letter}{Editoral Board\\Neural Computation}
\opening{Dear Sir or Madam:}

I am writing on behalf of myself and my co-authors, Matthew Scott and Chris Eliasmith, to present you with a manuscript that we have prepared for publication in Neural Computation, entitled ``Heterogeneity increases information transmission of neuronal populations''. This article examines the effects of heterogeneity on information transmission in populations of simulated neurons, and finds that it has a role to play in increasing the amount of information a population is able to encode and transmit about an input signal. We compare these results with previous results in the stochastic resonance field detailing how noise can benefit information transmission in neural populations, and find that heterogeneity has similar effects to noise in this respect.

This research is important to the field because it advances the view that heterogeneity has a role to play in neural systems, and explains in detail why this is so. Research in stochastic resonance has done a great service by documenting the beneficial effects of noise in simulated, in vitro, and in vivo neural systems, but this same service has not yet been extended to heterogeneity. The goal of this manuscript is to provide theoretical foundations for understanding the mechanisms by which heterogeneity operates, and why it is important to neural systems. These foundations tie together the limited amount of existing research on the subject, and provide direction and motivation for future theoretical and experimental research into heterogeneity in neural systems.

Given the general interest of this paper and its potential to inspire future research in the area, we ask that you consider the enclosed manuscript for publication in Neural Computation. None of the material in the article has been published or is under consideration for publication elsewhere. We thank you for your time, and look forward to your reply.

\closing{Sincerely,}
 
\end{letter}
\end{document}
